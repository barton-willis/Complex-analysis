\documentclass[12pt]{article}
%\documentclass[border=0.1cm]{standalone}
\usepackage{wasysym}
\usepackage{phonenumbers}
\usepackage{marvosym}
\usepackage{xcolor}
\usepackage[super]{nth}
\usepackage[paperwidth=8.5in,paperheight=11in,margin=0.45in]{geometry} 
\usepackage[USenglish]{babel}
\usepackage[USenglish]{isodate}% http://ctan.org/pkg/isodate
\usepackage[final]{pdfpages}
\usepackage{hyperref}
\hypersetup{
  colorlinks   = true, %Colours links instead of ugly boxes
  urlcolor     = black, %Colour for external hyperlinks
  linkcolor    = black, %Colour of internal links
  citecolor    = black %Colour of citations
}
\usepackage[activate={true,nocompatibility},final,tracking=true,kerning=true,spacing=true,factor=1100,stretch=10,shrink=10]{microtype}
\frenchspacing
\usepackage[nodayofweek,level]{datetime}
\usepackage{calc,url}
\newcounter{qz}\setcounter{qz}{0}
\newcommand{\qz}{%\
\setcounter{qz}{\value{qz}+1}
\textbf{In-class  \theqz} \,}

\newcounter{hw}\setcounter{hw}{0}
\newcommand{\hw}{%\
\setcounter{hw}{\value{hw}+1}
\textbf{HW \thehw}}

\newcounter{ex}\setcounter{ex}{0}
\newcommand{\ex}{%\
\setcounter{ex}{\value{ex}+1}
Exam \theex}

\usepackage[T1]{fontenc} 
\usepackage{fourier}
%\usepackage{tgschola} %to look retro
\newenvironment{mypar}[2]
  {\begin{list}{}%
    {\setlength\leftmargin{#1}
    \setlength\rightmargin{#2}}
    \item[]}
  {\end{list}}


\newcounter{wk}\setcounter{wk}{0}
\newcommand{\wk}{%\
\setcounter{wk}{\value{wk}+1}
\thewk \,\,}

\usepackage[nomessages]{fp}% http://ctan.org/pkg/fp


\usepackage{enumerate}
\usepackage{graphicx}

\usepackage{paralist}
\renewenvironment{description}[0]{\begin{compactdesc}}{\end{compactdesc}}

\newenvironment{alphalist}{
  \begin{enumerate}[(a)]
    \addtolength{\itemsep}{-0.75\itemsep}}
  {\end{enumerate}}
  \cleanlookdateon% Remove ordinal day reference
  \newcommand{\RomanNumeralCaps}[1]
      {\MakeUppercase{\romannumeral #1}}

\usepackage{xspace}
\makeatletter
\DeclareRobustCommand{\maybefakesc}[1]{%
  \ifnum\pdfstrcmp{\f@series}{\bfdefault}=\z@
    {\fontsize{\dimexpr0.8\dimexpr\f@size pt\relax}{0}\selectfont\uppercase{#1}}%
  \else
    \textsc{#1}%
  \fi
}
\newcommand\AM{\maybefakesc{am}\xspace}
\newcommand\PM{\maybefakesc{pm}\xspace}
\makeatother

 \newcommand{\coursename}{Complex Analysis}
\newcommand{\coursenumber}{MATH 365}
\newcommand{\sectionnumber}{01}
\newcommand{\term}{Spring }
\newcommand{\room}{Discovery Hall, room 386}
\newcommand{\meetingtime}{This class meets Monday, Wednesday, and Friday in \room \/  from 1:25\PM to 2:15 \PM.}
\newcommand{\officehours}{Monday, Wednesday, and Friday 10:00\AM-11:00\AM,
    Tuesday and Thursday 12:00 noon-2:00\PM, and by appointment.}

\begin{document}
\cleanlookdateon% Remove ordinal day reference
\shortdate
\printyearoff
\large
\begin{center}
    \textbf{\coursename}  \\
    {\coursenumber--\sectionnumber} \\
     {\term \the\year} \\
\end{center}

\vskip0.25in
\normalsize


\begin{center}
\begin{description}
    \item[Instructor:] Barton Willis, PhD, Professor of Mathematics
    \item[Office:]  Discovery Hall, Room 368
    \item[\phone:]   \phonenumber[country=US]{3088658868}
    \item[\Email:]    \href{mailto:willisb@unk.edu}{willisb@unk.edu}
    \item[Zoom for classes:] For Zoom class meetings, use the Meeting ID: 616 568 5706. 
    \item[Office Hours:] \officehours
  \end{description}
\end{center}

\subsubsection*{Class meeting time and place}

\meetingtime



\subsubsection*{Course Resources}

\noindent Our textbook is \emph{A First Course in Complex Analysis} edition 1.54,  by  
Beck, Marchesi, Pixon, and Sabalka.


\subsubsection*{Important Dates}

\begin{mypar}{0.25in}{0.25in} 

    \textbf{First Homework due} \dotfill  \textbf{\printdate{27/1/\the\year}}  \\
    \textbf{Exam 1} \dotfill \textbf{\printdate{17/2/\the\year}}  \\
    \textbf{Exam 2} \dotfill  \textbf{\printdate{24/3/\the\year}} \\
    \textbf{Exam 3} \dotfill \textbf{\printdate{21/4/\the\year}} \\
    \textbf{Final exam} \dotfill  \textbf{\printdate{17/5/\the\year}, 1:00 \PM--3:00 \PM}
\end{mypar}



\subsubsection*{Grading}

Your course grade will be based on weekly homework sets, class participation,
three midterm exams, and a comprehensive 
final exam; specifically:
\begin{mypar}{0.25in}{0.25in}
    \textbf{Weekly Homework}  \emph{11 fifteen point assignments}  \dotfill 165 (total) \\
    \textbf{Class participation} \dotfill 60 (total) \\
    \textbf{Mid-term exams 1,2, and 3} \emph{100 points each} \dotfill 300 (total)\\
    \textbf{Comprehensive Final exam} \dotfill 150 (total)
\end{mypar}
If it is necessary to adjust the number of  homework assignments,  your homework point 
total will be scaled to a total of 165.  For example, if we have only ten homework sets, 
your homework score will be scaled by a factor of \(165/150\). The same holds
for class participation.

\FPeval{\points}{round(165+60+300+150,0)}

\FPeval{\F}{round(\points*0.6-1,0)}
\FPeval{\Dm}{round(\points*0.6,0)}
\FPeval{\D}{round(\points*0.63,0)}
\FPeval{\Dp}{round(\points*0.66,0)}

\FPeval{\Cm}{round(\points*0.7,0)}
\FPeval{\C}{round(\points*0.73,0)}
\FPeval{\Cp}{round(\points*0.76,0)}

\FPeval{\Bm}{round(\points*0.8,0)}
\FPeval{\B}{round(\points*0.83,0)}
\FPeval{\Bp}{round(\points*0.86,0)}

\FPeval{\Am}{round(\points*0.9,0)}
\FPeval{\A}{round(\points*0.93,0)}
\FPeval{\Ap}{round(\points*0.98,0)}
The following table shows the \emph{minimum} number of points (out of \points) that
are required for each of the twelve letter grades D- through A+. For
example, a point total of \Bp\/  points will earn you a grade of B+,  and 
a point total of \Am\/ points will earn you a grade of A-. A point
total of \F\/  or less earns you a failing course grade.
 
 \vspace{0.1in}
     \begin{minipage}{5.5in}
  \centering 
\begin{mypar}{0.25in}{0.25in}
    \begin{minipage}{2.5in}
        D-  \dotfill \Dm \\
        D \dotfill \D \\
        D+ \dotfill \Dp \\
        C- \dotfill \Cm  \\
        C \dotfill \C \\
        C+ \dotfill \Cp 
        \end{minipage}
    \phantom{xxx}
    \begin{minipage}{2.5in}
        B- \dotfill \Bm \\
        B \dotfill  \B \\
        B+ \dotfill  \Bp\\
        A- \dotfill  \Am \\
        A \dotfill  \A \\
        A+ \dotfill  \Ap
    \end{minipage}
\end{mypar} 
\end{minipage}

\subsubsection*{Class participation}

We will have various activities for earning class participation points. You 
might be assigned a problem to present to the class, or we might work do 
in-class pair work.  Your class participation grade will be scaled to fifty
points.

\subsubsection*{Prerequisite}

The prerequisite for MATH 365 is an earned grade of D- or higher in 
MATH 260.

\subsubsection*{Catalog description}

\textbf{Complex Analysis (3 credit hours)}
Complex analysis is an introduction to the theory of complex variables and 
the calculus of analytic functions. Topics covered include the calculus of 
residues, the Cauchy Integration theorem, and the extension of 
exponential, logarithmic, and trigonometric functions to the complex plane.

\subsubsection*{Learning Outcomes}
On completion of this course, students will 
\begin{alphalist}
    \item be able to represent complex numbers algebraically and geometrically.
    \item be able to use the definition of the limit to prove that a function has a limit.
    \item be able to find derivatives of complex valued functions from the limit definition.
    \item understand the connection between the complex exponential and the trigonometric functions and be able to 
        prove trigonometric identities using these connections.
    \item be able to use the Cauchy-Riemann equations to determine if a function is 
          analytic.
    \item be able to represent functions using Laurent and power series and be able 
           to find function residues, poles, and pole order.
    \item be able to evaluate contour integrals using residues.
    \item understand the Cauchy integral formula and its consequences,  including the 
       fundamental theorem of algebra.
\end{alphalist}

\subsubsection*{Course Calendar}

Generally, we'll adhere to the scheduled exam dates even if we are ahead or behind with course work.  
When we are ahead or behind, the topics on the exams will be appropriately adjusted.  


\vspace{0.1in}
\noindent \textbf{Notices:}


\begin{alphalist}
   \item \emph{Exams will be given on the \textbf{Friday} of the week they are assigned.}
   

    \item Homework (\textbf{HW}) will be due at midnight on  Saturday of the week they are assigned.  


\end{alphalist}

\vspace{0.1in}

\begin{center}
    \small
\begin{tabular}  {|l|l|l|l|l|}
\hline
{\bf Week}  & \textbf{Week Starting} &  {\bf Section(s)} & {\bf Topic(s)} & \textbf{Assessment} \\
\hline \hline 
\wk    &  \printdate{23/1/\the\year} &    \S 1.1, \S 1.2  &  Algebraic Properties;  Geometric Properties  & \hw  \\
\wk    & \printdate{30/1/\the\year}   &  \S1.3, \S1.4 &  Geometric Properties; Topology & \hw  \\
\wk    & \printdate{6/2/\the\year}&     \S2.1, \S2.2  &   Limits \& continuity; Derivatives &  \hw \\
\wk    & \printdate{13/2/\the\year}   &     \S2.3, \S2.4  & Cauchy-Riemann; Constant functions   &   \textbf{\ex}       \\ \hline
\wk    & \printdate{20/2/\the\year} &  \S3.1, \S3.2    &  M\"obius transformations; Cross ratio   & \hw \\ 
\wk    & \printdate{27/2/\the\year}    & \S3.3, \S3.4, \S3.5  &  Exponential, trigonometric \& logarithmic functions &    \hw  \\
\wk    & \printdate{6/3/\the\year}     & \S4.1, \S4.2  &  Integration; Antiderivatives  & \hw \\
\wk    & \printdate{20/3/\the\year}   & \S4.3, \S4.4 &  Cauchy's Theorem \&  Cauchy's Integral Theorem  &   \textbf{\ex}   \\ \hline
\wk   &  \printdate{27/3/\the\year}   & \S5.1, \S5.2, \S5.3  &  Applications of Cauchy's Theorem \& Antiderivatives  & \hw \\ 
\wk   &  \printdate{3/4/\the\year}      &   \S6.1, \S6.2 &  Harmonic Functions; Mean-Value Theorems &   \hw \\
\wk   &  \printdate{10/4/\the\year}   &   \S7.1, \S7.2 & Sequences \& Series  & \hw  \\
\wk   & \printdate{17/4/\the\year}  & \S7.3, \S7.4 &  Sequences of functions; Regions of convergence   &   \textbf{\ex}  \\ \hline
\wk   & \printdate{24/4/\the\year} & \S8.1, \S8.2 &   Power series& \hw \\
\wk   & \printdate{1/5/\the\year}    &  \S8.3, \S10.1,  &  Classification of zeros; Infinite sums &  \hw   \\
\wk   & \printdate{8/5/\the\year}   &  \S10.2, \S10.3    & Binomial coefficients; Fibonacci numbers  &      \\
 \wk   & \printdate{15/5/\the\year}     &  &    \hfill  & \textbf{ Final Exam}  \\  \hline
   
\end{tabular}
\end{center}



\subsubsection* {Policies}

Unless an assessment is \emph{explicitly} stated to be a group project,  \emph{all work you turn in for a grade must be your own.}  If you need assistance in completing a homework assignment, you may ask me for help. Googling for answers, seeking help from the Learning Commons or other faculty members,  or using solution keys from previous terms (either from UNK or other universities) is also prohibited.  Violation of these rules will result in earning a grade of zero on the assessment. Each homework assignment you turn in for a grade must include the statement:

\begin{quote}
\fbox{I have neither given nor received unauthorized assistance on this assignment.}
\end{quote}
 If two assignments are so similar that only collaboration could explain their similarities, both assignments will receive a grade of zero.  Using unauthorized materials or communication devices (cell phone, for
example) while taking a test will earn you a grade of zero on that assessment.  
For the university academic integrity policy, please read
\small \url{https://catalog.unk.edu/undergraduate/academics/academic-regulations/academic-integrity-policy/}
 
\normalsize
Specially, our course policies are:
\begin{enumerate}

\item Regular in person class attendance is required. If you are ill or need to miss 
class due to athletics, please let me know ahead of time and I will make an effort to put the class on Zoom. 
Our classroom technology often doesn't work, so do not rely on watching recorded classes.

\item There is no explicit grade penalty for not attending class. But if you choose to not attend class for reasons other
than illness or athletics, I reserve the right to not be all that helpful in giving you assistance on homework or helping 
you learn missed material.

\item All examinations, including the final exam, must be taken in person.

\item For examinations and in class assignments, show your work.  \emph{No credit will be given for multistep problems without the necessary work. Your solution must contain enough detail
so that I am convinced that you could correctly work any similar problem.} Also erase or clearly mark any work you want me to ignore; otherwise,
I'll grade it.  

\item The work you turn in is expected to be \emph{accurate, 
complete, concise, neat}, and \emph{well-organized}.  
\emph{You will not earn full credit on work that falls short of 
these expectations.}

\item Class cancellations due to weather, illness, or other 
unplanned circumstances may require that we make  adjustments
to the course calendar, exam dates, due dates, or specifics for 
course assessments. 


\item Extra credit is not allowed. 



\item For examinations, you may use a teacher provided quick reference sheet, 
but no other reference materials. You may also use a pencil, eraser, 
and a scientific calculator. For examinations, your phone and all such
devices must be turned off and \emph{out of sight}. 

\item Generally, if you are ill or absent for any reason (including 
athletics), you must turn in your in class work on time. Permission to
turn in work late must be made before the due date, otherwise late in class work 
will count zero points.

\item During class time, please refrain from using electronic devices. If your 
device usage distracts your classmates, I will ask you to put it away. If it's my 
impression that you are often not paying attention in class, I reserve the right to 
decline to help you during office hours.

\item The final examination will be \emph{comprehensive} and it will be given 
during the  time scheduled by the University. Except for \emph{extraordinary circumstances}
you must take the exam at this time.
 
\item If you have questions about how your work has been graded, make an appointment with me immediately.


\item Please regularly check Canvas  to verify that your scores have 
been recorded correctly.  If I made a mistake in recording one of
your grades, I'll correct it provided you saved your paper.

\item The course calendar might be modified. It is your responsibility to attend
class and to keep up to date on modifications to the course calendar.

\end{enumerate}

\newpage

\subsubsection*{Reporting Student Sexual Harassment, Sexual Violence or Sexual Assault}

Reporting allegations of rape, domestic violence, dating violence, sexual assault, sexual harassment, and stalking enables the University to promptly provide support to the impacted student(s), and to take appropriate action to prevent a recurrence of such sexual misconduct and protect the campus community. Confidentiality will be respected to the greatest degree possible. Any student who believes they may be the victim of sexual misconduct is encouraged to report to one or more of the following resources:
\begin{description}
    \item[Local Domestic Violence, Sexual Assault Advocacy Agency]\phonenumber[country=US]{3082372599}

    \item[Campus Police (or Security)]\phonenumber[country=US]{3088658911}
    
    \item[Title IX Coordinator]\phonenumber[country=US]{3088658655}

\end{description}
Retaliation against the student making the report, whether by students or University employees, will not be tolerated.

\subsubsection*{Students with Disabilities}

It is the policy of the University of Nebraska at Kearney to provide flexible and individualized reasonable accommodation to students with documented disabilities. To receive accommodation services for a disability, students must be registered with the UNK Disabilities Services for Students (DSS) office, 175 Memorial Student 
Affairs Building,  \phonenumber[country=US]{3088658214} or by 
email \href{mailto:unkdso@unk.edu}{unkdso@unk.edu}

\subsubsection*{Students Who are Pregnant}

It is the policy of the University of Nebraska at Kearney to provide flexible 
and individualized reasonable accommodation to students who are pregnant. 
To receive accommodation services due to pregnancy, students must contact 
the Student Health office at \phonenumber[country=US]{3088658218}. The following 
links provide information for students and faculty regarding pregnancy 
rights:
\small
\begin{enumerate}
  \item \url{https://thepregnantscholar.org/title-ix-basics/}
  \item \url{https://nwlc.org/resource/faq-pregnant-and-parenting-college-
  graduate-students-rights/UNK Statement of Diversity & Inclusion}
\end{enumerate}
\normalsize

\subsubsection*{UNK Statement of Diversity \& Inclusion}

UNK stands in solidarity and unity with our students of color, our Latinx and international students, our LGBTQIA+ students 
and students from other marginalized groups in opposition to racism 
and prejudice in any form, wherever it may exist. It is the job of 
institutions of higher education, indeed their duty, to provide a 
haven for the safe and meaningful exchange of ideas and to support 
peaceful disagreement and discussion. In our classes, we strive to 
maintain a positive learning environment based upon open 
communication and mutual respect. UNK does not discriminate on the 
basis of race, color, national origin, age, religion, sex, gender, 
sexual orientation, disability or political affiliation. 
Respect for the diversity of our backgrounds and varied life 
experiences is essential to learning from our similarities as well 
as our differences. The following link provides resources and other 
information regarding D\&I: \url{https://www.unk.edu/about/equity-access-diversity.php}

\includepdf[pages={1-},angle=90]{door-schedule.pdf}
\includepdf[pages={1-},angle=90]{complex-analysis-quick-reference.pdf}

\end{document}

