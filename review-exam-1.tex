\documentclass[fleqn,12pt]{exam}
\usepackage{pifont,enumerate,url}
\usepackage{dingbat}
\usepackage{amsmath}
\usepackage{fleqn}
\usepackage{epsfig,upgreek}
\usepackage{mathptm,color}
\newenvironment{handlist}{
  \begin{enumerate}[\leftthumbsup]
    \addtolength{\itemsep}{-1.0\itemsep}}
  {\end{enumerate}}

\addpoints
\boxedpoints
\pointsinmargin
\pointname{pts}

\newcommand{\dotprod}{\, {\scriptzcriptztyle \stackrel{\bullet}{{}}}\,}
\newcommand{\complex}{\mathbf{C}}
\newcommand{\integers}{\mathbf{Z}}
\newcommand{\nat}{\mathbf{N}}
\newcommand{\imag}{\mathrm{i}}
\newcommand{\range}{\mathrm{range}}
\newcommand{\Arg}{\mathrm{Arg}}
%\usepackage[euler-digits,euler-hat-accent,T1]{eulervm}
\usepackage{fourier}
%\definecolor{SolutionColor}{rgb}{0.9,1,1}
\definecolor{SolutionColor}{rgb}{1,1,0.7}
\addpoints
\boxedpoints
\pointsinmargin
\pointname{pts}
\newcommand{\reals}{\mathbf{R}}

\begin{document}
\large
\vspace{0.1in}
\noindent\makebox[3.0truein][l]{{\bf MATH 365}}
{\bf Name:}\hrulefill\
\noindent \makebox[3.0truein][l]{\bf Review for Exam  1}
%{\bf Row and Seat:}\hrulefill\
\normalsize


\begin{questions}

\question In the context of complex numbers, know the definitions of a \emph{neighborhood, open set, closed set, connected set, bounded set}.
\question  Find the \emph{rectangluar form} of \(\sqrt{1 + \imag} \).







%\question  Explain why that for all \(z \in \complex\), we have \(\Arg(\sqrt{z}) \in \left(-\frac{\uppi}{2}, \frac{\uppi}{2}\right )\).

\begin{solution}%[1.5in]

\end{solution}


\question  Explain why the solution set of  \(\sqrt{z} = -2019 + \imag\) is empty.


\begin{solution}%[1.5in]

\end{solution}

\question  Using the \emph{limit of a Newton quotient}, find the derivative of \(z \in \complex \mapsto z^2\) at \(\imag\).

\begin{solution}%[2.0in]
\[
  \lim_{x \to 5} \frac{x^2 - x - (5^2 - 5)}{x - 5} = \lim_{x \to 5} \frac{x^2 - x - 20}{x-5} =  \lim_{x \to 5} \frac{(x-5)(x+4)}{x-5} = 
\lim_{x \to 5} (x+4) = 5+4=9.
\]

\end{solution}

\question    Use the Cauchy Riemann (CR) equations to show that the function \(z \in \mathbf{C} \mapsto z + \overline{z}\)
is not differentiable anywhere on \(\mathbf{C}\). 
\begin{solution}%[2.5in]

\end{solution}



\question Find the \emph{rectangular representation} for each complex number; that is express each
complex number  \emph{explicitly} in the form \(a + i b\), where \(a,b \in \reals\).

\begin{parts}

\part   \((5  + \imag) (2 + \imag) = \)
\begin{solution}%[1.2in]
\[\left( 5-i\right) \,\left( i+7\right) = 35 + 5 i - 7 i - i^2 = 35 - 2\, i + 1 = 36-2\,i\]
\end{solution}

\part   \(\frac{2 + \imag}{2 -  \imag} =  \)

\begin{solution}%[1.2in]
\[\frac{i+2}{i+4}= \frac{i+2}{i+4} \frac{\overline{i+4}} {{\overline{i+4}}} = \frac{2 i + 9}{17} =  \frac{2\,i}{17}+\frac{9}{17}. \]
\end{solution}
\part    \((2 + \imag) \overline{(3  + \imag)} = \)

\begin{solution}%[1.5in]
\[\left(2 +  \imag \right) \,\overline{(3 + \imag)} =  \left( i+2\right) \, (3+i) =  5\,i+5\]
\end{solution}

\end{parts}


%\newpage

\question  Using the definition of the limit,  show  that \(\displaystyle \lim_{z \to \imag} z^2 = -1\).

%---------
\question Find the \emph{rectangluar form} for each square root
\begin{parts}

 \part   \(\sqrt{\frac{5\,\sqrt{3}\,\imag}{2}+\frac{5}{2}}\)

\begin{solution} We have
\[
   \frac{5\,\sqrt{3}\,\imag}{2}+\frac{5}{2} = 5\,{e}^{\frac{i\,\pi }{3}} .
\]
So
\[
    \sqrt{\frac{5\,\sqrt{3}\,\imag}{2}+\frac{5}{2}} = \sqrt{5} {e}^{\frac{i\,\pi }{6}}  = \frac{\sqrt{5}\,i}{2}+\frac{\sqrt{3}\,\sqrt{5}}{2}.
\]
\end{solution}

 \part  \(\sqrt{-\imag}\)
\end{parts}


\begin{solution}
\[
   \sqrt{-\imag} = \frac{1}{\sqrt{2}}-\frac{i}{\sqrt{2}}.
\] 
\end{solution}

%\question  Find the \emph{range} of the cube root function. Draw a \emph{labeled} picture of the range.

%\begin{solution} For \(R \in \reals_{\geq 0} \) and \(\theta \in (-\pi,\pi]\), we have
%\[
%  \sqrt[3]{R e^{i \theta}} = \sqrt[3]{R} e^{i \theta / 3}.
%\]
%Since \(\theta \in (-\pi,\pi]\), we have \(\theta /3 \in (-\pi/3,\pi/3]\). The magnitude of a cube root has
%no restriction, but the argument of a cube root must be in the interval \((-\pi/3,\pi/3]\). Graphically the
%range of the cube root function is a wedge.
%\end{solution}

%\question  Solve the equation \(\sqrt[3]{z} = -1 + i\). Justify your answer.

%\begin{solution}
%Since \(\mbox{arg} ( -1 + i) \notin (-\pi/3,\pi/3]\), the number \(-1 + i\) isn't in the range of the cube root. Thus the solution %set
%is empty. 

%\textbf{TE} Knowing something about the range of a function really does matter. In the context of solving equations,
%it's essential. The general fact is that if \(y \notin \range(F)\), the solution set of the equation \(F(x) = y\) is
%\empty{empty}. An example is the equation
%\[
 % \left | x + \left|x - 8 \right | - 42 \right | = -1.
%\]
%Quick: what's the solution set? Since -1 isn't in the range of the absolute value function, the solution set is empty.

 %Remember that squaring or cubing an equation sometimes makes the solution set larger. To illustrate,
%the solution set of \(x=1\) is \(\{1\}\), but the solution set of \(x^2 = 1^2\) is \(\{-1,1\}\).
%Indeed for our case
%\[
 % \{z \in \complex \, | \,  \sqrt[3]{z} = -1 + i \} \neq \{z \in \complex \, | \,  z = (-1 + i)^3 \}
%\]


%\end{solution}

%\question  Show that the square root and the conjugate do not commute; that is,  show that 
%\mbox{\( \sqrt[2] {\overline{z}} = \overline{\sqrt[2] {z}} \)} isn't an identity.

%\begin{solution} We have \(\sqrt[2] {\overline{-1}} = \sqrt[2] {-1} = \imag\). But \(\overline{\sqrt[2] -1} = \overline{\imag} = -\imag \)

%``Singularity is almost invariably a clue.'' ({\sc Sherlock}).
%\end{solution}

%\question  Find the \emph{rectangular} form for \( \left(\frac{\sqrt{3}}{2} + \frac{i}{2} \right)^{2014} \). The crazy
%range reducing function \(\mathcal{S}\) might be useful; it is
%\(
%   \mathcal{S} = x \in \reals \mapsto x - 2 \pi \left \lceil \frac{x - \pi}{2 \pi} \right \rceil.
%\)  

%\begin{solution}
%\[{\left( \frac{i}{2}+\frac{\sqrt{3}}{2}\right) }^{2014}=\frac{1}{2}-\frac{\sqrt{3}\,i}{2}\]
%\end{solution}

\question  Show that the function \(z \in \complex \mapsto \overline{z}  \) is \emph{continuous} at 0.

\begin{solution} (See classnotes.)

\end{solution}







\question   Use the Cauchy Riemann (CR) equations to show that the function \(z \in \mathbf{C} \mapsto z - \overline{z}\)
is not differentiable anywhere on \(\mathbf{C}\). 
\begin{solution}%[2.5in]

\end{solution}




\question  Using the exponential form for the trigonometric functions, show that
\[ 
   \mathrm{cos}\left( x\right) \,\mathrm{sin}\left( x\right) =\frac{\mathrm{sin}\left( 2\,x\right) }{2}.
\]
The exponential form for cosine is \(\cos(x) = \frac{e^{i x} + e^{-i x}}{2}\).


\begin{solution}%[3.8in]
\end{solution}


\question Use the definition of continuity (from the classnotes--the one with the \(\varepsilon\) and \(\delta\) in it) 
to show that the function \(z \in \mathbf{C} \mapsto z^2\) is continuous at \(i\).

\begin{solution}%[3.8in]
\end{solution}

%\newpage



\question  Show that \(\sqrt{z w} = \sqrt{z} \sqrt{w} \) is \emph{not} an identity for complex \(z\) and \(w\). 
Here the square root is the principal square root.

\begin{solution}
We have \(\sqrt{-1 \times -1} = \sqrt{1} = 1\). But \(\sqrt{-1} \sqrt{-1} = \imag \times \imag = -1\).

\end{solution}










\question Know everything about complex number arithmetic; know what the commutative and associative
properties are; know how to find the rectangular form of sums, products, differences, and quotients
of complex numbers.
 
\question Know what the conjugate is and know its properties.


\question Know the definition of the argument of a complex number.

\question Know how to use \(\exp(i x) = \cos(x) + i \sin(x)\) along with the rule of
exponents to derive trigonometric identities (mostly classnotes).

\question Know how the principal square root of a complex number is
defined;

\question Know the triangle inequality.


\question Know the derivation of the CR equations

\question Know the defintion of continuity  for a \(\complex \to \complex\) function.











\question  Use \(\exp(i x)  = \cos(x) + i \sin(x)\) to show that
\[
 {\mathrm{cos}\left( x\right) }^{3}-3\,\mathrm{cos}\left( x\right) \,{\mathrm{sin}\left( x\right) }^{2}=\mathrm{cos}\left( 3\,x\right) 
\]
is an identity. More esthetically, the identity is \(4\,{\mathrm{cos}\left( x\right) }^{3}-3\,\mathrm{cos}\left( x\right) =\mathrm{cos}\left( 3\,x\right) \).




\question  Find the \emph{polar} representation for the number \(-1 + i\). Use the 
polar form to find the \emph{polar} representation for \(\sqrt{-1+i}\).
\begin{solution}%[1.2in]
\[i-1=\sqrt{2}\,{e}^{\frac{3\,i\,\pi }{4}}\]
Also
\[\sqrt{-1+i} = 2^{1/4} {e}^{\frac{3\,i\,\pi }{8}}\]
\end{solution}



\question  Using the rule of exponents along with \(\exp(i x) = \cos(x) + i \sin(x)\) to 
show that \(\cos(2 x) = \cos^2(x) - \sin^2(x)\) is an identity.
\begin{solution}%[2.5in]
\[
 \exp(2 i x) = \cos(2x) + i \sin(2x) = (\cos(x) + i \sin(x))^2 = \left(\cos^2(x) - \sin^2(x)\right) + 2 i
  \cos(x) \sin(x).
\]
So \(\cos(2 x) = \cos^2(x) - \sin^2(x)\) is an identity.
\end{solution}











\question Know the triangle inequality.






\question  For \(x \in \reals\), we have the identities \(\sin(x) = \frac{e^{i x} - e^{-i x}}{2 i}\) and \(\cos(x) =  \frac{e^{i x} + e^{-i x}}{2} \).
Use these facts along with facts about the complex exponential function to  show that
\[
   {\mathrm{sin}\left( x\right) }^{3}=-\frac{\mathrm{sin}\left( 3\,x\right) -3\,\mathrm{sin}\left( x\right) }{4}
\]
is an identity.

\begin{solution} We have
\begin{align}
  \sin^3(x) &= \frac{{\left( {e}^{i\,x}-{e}^{-i\,x}\right) }^{3}}{- i 8}, \\
            &= \frac{i\,{e}^{3\,i\,x}}{8}-\frac{3\,i\,{e}^{i\,x}}{8}+\frac{3\,i\,{e}^{-i\,x}}{8}-\frac{i\,{e}^{-3\,i\,x}}{8}, \\
            &= \frac{3\,\mathrm{sin}\left( x\right) }{4}-\frac{\mathrm{sin}\left( 3\,x\right) }{4}.
\end{align}

\end{solution}

\question  Find the \emph{rectangular} form for \( \left(\frac{\sqrt{3}\,i}{2}+\frac{1}{2} \right)^{107} \).




\end{questions}
\end{document}

\noindent \textbf{Claim} The sequence  \(F = k \in \integers_{>1} \mapsto \sum_{\ell = 1}^k \frac{1}{\ell^2} \) is Cauchy.

To show this, we start with a calculation. Suppose \(k > l > N > 1\). Then
\begin{align}
  | F_k - F_l | &= | \sum_{s=l+1}^k \frac{1}{s^2} |, & \mbox{(additivity of sums)} \\
                &= \sum_{s=l+1}^k \frac{1}{s^2}, & \mbox{(everybody is positive)} \\
                &<  \sum_{s=l+1}^k \frac{1}{s (s-1)}, &\mbox{(trick)}  \\
                &= \sum_{s=l+1}^k \left( \frac{1}{s-1}-\frac{1}{s} \right) & \mbox{(partial fraction)} \\
                &= \frac{1}{l} - \frac{1}{k}, & \mbox{(massive cancellation)} \\
                & <  \frac{1}{l}, & \mbox{(bigger by dropping negativity)} \\
                & <  \frac{1}{N}, & \mbox{(make the denominator  smaller)} 
\end{align}
We need \(\frac{1}{N} \leq  \varepsilon\). So all we need is to take \(N = \lceil \frac{1}{\varepsilon} \rceil \).
\begin{questions}


\question  Over the reals, the factorization of \({z}^{3}-1\) is \((z-1)({z}^{2}+z+1) \).
Since \({z}^{2}+z+1 \) has no real roots, no further factorization is possible. Again, sticking
to the real numbers, the partial fraction decomposition of \(\frac{1}{{z}^{3}-1}\) is given by
\[
    \frac{1}{{z}^{3}-1}=\frac{1}{3\,\left( z-1\right) }-\frac{z+2}{3\,\left( {z}^{2}+z+1\right) }.
\]
In the previous assignment, you discovered that over the complex numbers \(z^3-1\) factors into the 
product of three linear terms; specifically,
\[ 
   z^3-1 = \left( z-1\right) \,\left( z-\frac{\sqrt{3}\,i-1}{2}\right) \,\left( z+\frac{\sqrt{3}\,i+1}{2}\right).
\]
This suggests the possibility that over the complex numbers, the partial fraction decomposition of \(\frac{1}{z^3-1} \)
has the form
\[
  \frac{1}{z^3 -1} = \frac{A}{z-1} + \frac{B}{ z-\frac{\sqrt{3}\,i-1}{2}} + \frac{C}{ z+\frac{\sqrt{3}\,i+1}{2}},
\]
where \(A,B, C \in \mathbf{C}\). Show that there are such numbers \(A,B, C\).
To do this calculation, it might be easiest to substitute 
\[
  \alpha \leftarrow \frac{\sqrt{3}\,i-1}{2}, \beta \leftarrow -\frac{\sqrt{3}\,i+1}{2}\,
\]
and use the identities
\(
   {\alpha}^{2}=-\alpha-1,\alpha\,\beta=1, \, {\beta}^{2}=\alpha, \, \beta + \alpha  = -1.
\)

\begin{solution} We need to find \(A,B\) and \(C\) such that
\[
  1 = A (z-\alpha) (z-\beta) + B  (z-1)(z-\beta) + C (z-1) (z-\alpha)
\]
is an identity. Substituting each of  \(z \leftarrow 1\), \(z \leftarrow \alpha\), and \(z \leftarrow \beta\) yields the
three decoupled equations for \(A,B\) and \(C\):
\[
  \begin{cases}  1 &= A (1-\alpha) (1-\beta) \\ 1 &= B (\alpha-1) (\alpha-\beta) \\ 1 &= C (\beta - 1) (\beta - \alpha) \end{cases}
  = \begin{cases}  1 &= 3 A, \\ 1 &= 3 \beta B \\ 1 &= 3 \alpha C  \end{cases} = 
   \begin{cases} A &= \frac{1}{3} \\ B &= \frac{1}{3 \beta} \\ C &= \frac{1}{3 \alpha} \end{cases}
\]
We now know that for three distinct values for \(z\) (specifically for \(z = 1, z = \alpha\), and \(z = \beta\)) that
\[
  1 = A (z-\alpha) (z-\beta) + B  (z-1)(z-\beta) + C (z-1) (z-\alpha).
\]
But we need to know that equality holds not just for these three values for \(z\), but for all values of \(z\). We could expand
the right side and show that it crunches to 1. Alternatively, we could use the fact that a second degree polynomial is uniquely 
determined by three distinct points on its graph--this works in the real case and in the complex case.

\textbf{TE} The polynomials \(z \in \mathbf{C}  \mapsto z -1\), \(z \in \mathbf{C}  \mapsto z - \alpha\),  and
\(z \in \mathbf{C}  \mapsto z - \beta\) have no common factors, so they are said to be \emph{relatively prime.} Alternatively
(gcd = greatest common divisor)
\[
  \mbox{gcd}(z \in \mathbf{C}  \mapsto z -1, z \in \mathbf{C}  \mapsto z - \alpha, z \in \mathbf{C}  \mapsto z - \beta ) = 1
\]
We've shown that the greatest common divisor of these three polynomials is a linear combination of the three polynomials.
This isn't a coincidence, by the way; see, for example \url{http://en.wikipedia.org/wiki/Extended_Euclidean_algorithm}.
\end{solution}

\question  Find the \emph{range} of the cube root function. Draw a \emph{labeled} picture of the range.

\begin{solution} For \(R \in \reals_{\geq 0} \) and \(\theta \in (-\pi,\pi]\), we have
\[
  \sqrt[3]{R e^{i \theta}} = \sqrt[3]{R} e^{i \theta / 3}.
\]
Since \(\theta \in (-\pi,\pi]\), we have \(\theta /3 \in (-\pi/3,\pi/3]\). The magnitude of a cube root has
no restriction, but the argument of a cube root must be in the interval \((-\pi/3,\pi/3]\). Graphically the
range of the cube root function is a wedge.
\end{solution}

\question  Solve the equation \(\sqrt[3]{z} = -1 + i\). Justify your answer.

\begin{solution}
Since \(\mbox{arg} ( -1 + i) \notin (-\pi/3,\pi/3]\), the number \(-1 + i\) isn't in the range of the cube root. Thus the solution set
is empty. 

\textbf{TE} Knowing something about the range of a function really does matter. In the context of solving equations,
it's essential. The general fact is that if \(y \notin \range(F)\), the solution set of the equation \(F(x) = y\) is
\empty{empty}. An example is the equation
\[
  \left | x + \left|x - 8 \right | - 42 \right | = -1.
\]
Quick: what's the solution set? Since -1 isn't in the range of the absolute value function, the solution set is empty.

 Remember that squaring or cubing an equation sometimes makes the solution set larger. To illustrate,
the solution set of \(x=1\) is \(\{1\}\), but the solution set of \(x^2 = 1^2\) is \(\{-1,1\}\).
Indeed for our case
\[
  \{z \in \mathbf{C} \, | \,  \sqrt[3]{z} = -1 + i \} \neq \{z \in \mathbf{C}  \, | \,  z = (-1 + i)^3 \}
\]


\end{solution}

\question  Find a simple condition on the arguments of \(z_1\) and \(z_2\) that makes \(\sqrt[3]{z_1 z_2} = \sqrt[3]{z_1} \sqrt[3]{z_2}.\)
The simple condition should look something like \(\arg(z_1) + \arg(z_2) \in \mbox{some interval}\).
Also, find complex numbers \(z_1\) and \(z_2\) such that \(\sqrt[3]{z_1 z_2} \neq \sqrt[3]{z_1} \sqrt[3]{z_2}.\)

\begin{solution} For \(R_1, R_2 \in \reals_{\geq 0}\) and \(\theta_1, \theta_2 \in (-\pi, \pi]\), we have
\[
  \sqrt[3]{R_1 e^{i \theta_1}} \sqrt[3]{R_2 e^{i \theta_2}} = \sqrt[3]{R_1} e^{i \theta_1/3} \sqrt[3]{R_2} e^{i \theta_2/3}
     = \sqrt[3]{R_1 R_2} e^{i (\theta_1 + \theta_2)/3}.
\] 
For positive real numbers (specifically \(R_1\) and \(R_2\)), we used the famous fact that for nonnegative numbers, 
the product of the cube roots is the cube root of the product.

Now let's examine \(\sqrt[3]{R_1 e^{i \theta_1} {R_2 e^{i \theta_2}}}\). This time, we need to be careful because
\(\theta_1 + \theta_2\) is not automatically in the interval \((-\pi, \pi]\). To fix this, we need the crazy sawtooth
function
\[
  \mathcal{A} = x \in \reals \mapsto x - 2 \pi \left \lceil \frac{x - \pi}{2 \pi} \right \rceil.
\]
Then 
\[ 
    \sqrt[3]{R_1 e^{i \theta_1} {R_2 e^{i \theta_2}}} =  \sqrt[3]{R_1 R_2 e^{i \mathcal{A} (\theta_1 + \theta_2)}} = 
       \sqrt[3]{R_1 R_2}  e^{i \mathcal{A} (\theta_1 + \theta_2)/3} 
\]
For equality between \(\sqrt[3]{R_1 e^{i \theta_1}} \sqrt[3]{R_2 e^{i \theta_2}} \) and \(\sqrt[3]{R_1 e^{i \theta_1} {R_2 e^{i \theta_2}}}\),
we need
\[
   \frac{\theta_1 + \theta_2}{3} = \frac{1}{3} \mathcal{A} \left(\theta_1 + \theta_2 \right)  - 2 \pi k,
\]
where \(k \in \integers\). Using the formula for \(\mathcal{A}\) and solving for the term that involves the ceiling function gives
\[
  \left \lceil \frac{\theta_1 + \theta_2 - \pi}{2 \pi} \right \rceil = 3 k
\]
So \(\frac{\theta_1 + \theta_2 - \pi}{2 \pi} \) must round up to a multiple of three. So
\[
   \frac{\theta_1 + \theta_2 - \pi}{2 \pi} \in \dots \cup (-4,-3] \cup (-1,0] \cup (2,3] \cup (5,6] \cup \dots 
\]
Solving for  \( \theta_1 + \theta_2 \) gives
\[
   \theta_1 + \theta_2 \in \dots \cup (-7 \pi, -5 \pi] \cup (-\pi, \pi] \cup (5 \pi, 7 \pi] \cup (11 \pi, 13 \pi] \cup.
\]
But \(\theta_1 + \theta_1 \in (-2 \pi, 2 \pi]\), so the condition simplifies to 
\(
    \theta_1 + \theta_2 \in (-\pi,\pi]
\).

\quad To illustrate, we have
\[
  \sqrt[3]{(-1) (-1)} = \sqrt[3]{1} = \sqrt[3]{1} e^{i \frac{0}{3}} = 1.
\]
But
\[
  \sqrt[3]{-1} \sqrt[3]{-1} = \sqrt[3]{1} e^{i \pi /3}  \sqrt[3]{1} e^{i \pi /3} = e^{2 \pi i  /3} = \frac{\sqrt{3}\,i}{2}-\frac{1}{2}.
\]
Thus \(\sqrt[3]{(-1) (-1)} \neq \sqrt[3]{-1} \sqrt[3]{-1} \). This isn't surprising since
\(
   \mbox{arg}(-1) + \mbox{arg}(-1) = \pi + \pi  = 2 \pi \notin (-\pi,\pi]
\). An example where the cube root of the product is the product of the cube roots is:
\[
  \sqrt[3]{1+i} \sqrt[3]{1-i} = \sqrt[3]{\sqrt{2}}   e^{3 \pi i / 4}  \sqrt[3]{\sqrt{2}} e^{-3 \pi i / 4} = \sqrt[3]{\sqrt{2}} \sqrt[3]{\sqrt{2}} = \sqrt[3]{2},
\]
and
\(
  \sqrt[3]{(1+i)(1-i)} = \sqrt[3]{2}
\).
For this example, our criteria is satisfied: we have
\[
  \mbox{arg}(1-i) + \mbox{arg}(1+i) = \frac{3 \pi}{4} - \frac{3 \pi}{4} = 0 \in (-\pi,pi].
\]

\end{solution}
\question  For \(x \in \reals\), we have the identities \(\sin(x) = \frac{e^{i x} - e^{-i x}}{2 i}\) and \(\cos(x) =  \frac{e^{i x} + e^{-i x}}{2} \).
Use these facts along with facts about the complex exponential function to  show that
\[
   {\mathrm{sin}\left( x\right) }^{3}=-\frac{\mathrm{sin}\left( 3\,x\right) -3\,\mathrm{sin}\left( x\right) }{4}
\]
is an identity.

\begin{solution} We have
\begin{align}
  \sin^3(x) &= \frac{{\left( {e}^{i\,x}-{e}^{-i\,x}\right) }^{3}}{- i 8}, \\
            &= \frac{i\,{e}^{3\,i\,x}}{8}-\frac{3\,i\,{e}^{i\,x}}{8}+\frac{3\,i\,{e}^{-i\,x}}{8}-\frac{i\,{e}^{-3\,i\,x}}{8}, \\
            &= \frac{3\,\mathrm{sin}\left( x\right) }{4}-\frac{\mathrm{sin}\left( 3\,x\right) }{4}.
\end{align}

\end{solution}

\end{questions}
\end{document}

\question  Work Exercise 1.2 on page 4.

\question  Show that the sequence \(k \in \mathbf{N} \mapsto k\) diverges.

\begin{solution} Let \(L \in \reals\) and \(M \in N\); we'll show that there
is \(k \in M \dots \infty\) such that \(k \geq L + \frac{1}{2}\). Define
\(k = \mbox{max} \{ \lceil L + \frac{1}{2} \rceil, M \}\). Then \(k \in M \dots \infty\); and
further
\begin{align*}
  k &=  \mbox{max} \{ \lceil L + \frac{1}{2} \rceil, M \}, \\
    &\geq \lceil L + \frac{1}{2} \rceil, \\
    &\geq L + \frac{1}{2}.
\end{align*}
\begin{handlist}
\item \(|F_k - L | \geq \varepsilon \) is equivalent to \(F_k \geq L + \varepsilon\) or
\(F_k \leq  L - \varepsilon\). For this problem \(F_k = k\) and \( \varepsilon = \frac{1}{2}\).
\end{handlist}

\end{solution}

\end{questions}

\end{document}