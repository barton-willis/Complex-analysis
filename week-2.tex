\documentclass[12pt,fleqn,answers]{exam}
\usepackage{pifont}
\usepackage{dingbat}
\usepackage{amsmath,amssymb}
\usepackage{epsfig}
\usepackage{upgreek}
\usepackage[super]{nth}
\usepackage[colorlinks=true,linkcolor=black,anchorcolor=black,citecolor=black,filecolor=black,menucolor=black,runcolor=black,urlcolor=black]{hyperref}
\usepackage[letterpaper, margin=0.75in]{geometry}
\addpoints
\boxedpoints
\pointsinmargin
\pointname{pts}

\usepackage[activate={true,nocompatibility},final,tracking=true,kerning=true,factor=1100,stretch=10,shrink=10]{microtype}
\usepackage[american]{babel}
\usepackage[T1]{fontenc}
\usepackage{fourier}
\usepackage{isomath}
\usepackage{upgreek,amsmath}
\usepackage{amssymb}
\usepackage{graphicx}

\newcommand{\dotprod}{\, {\scriptzcriptztyle\stackrel{\bullet}{{}}}\,}

\newcommand{\reals}{\mathbf{R}}
\newcommand{\lub}{\mathrm{lub}} 
\newcommand{\glb}{\mathrm{glb}} 
\newcommand{\complex}{\mathbf{C}}
\newcommand{\dom}{\mbox{dom}}
\newcommand{\range}{\mbox{range}}
\newcommand{\cover}{{\mathcal C}}
\newcommand{\integers}{\mathbf{Z}}
\newcommand{\vi}{\, \mathbf{i}}
\newcommand{\vj}{\, \mathbf{j}}
\newcommand{\vk}{\, \mathbf{k}}
\newcommand{\bi}{\, \mathbf{i}}
\newcommand{\bj}{\, \mathbf{j}}
\newcommand{\bk}{\, \mathbf{k}}
\DeclareMathOperator{\Arg}{\mathrm{Arg}}
\DeclareMathOperator{\Ln}{\mathrm{Ln}}
\newcommand{\imag}{\, \mathrm{i}}

\usepackage{graphicx}
\usepackage{color}
\shadedsolutions
\definecolor{SolutionColor}{rgb}{0.8,0.9,1}
\newcommand\AM{\textsc{am}}
\newcommand\PM{\textsc{pm}}
     
\newcommand{\quiz}{2}
\newcommand{\term}{Spring}
\newcommand{\due}{Friday 3 Feb \PM}
\newcommand{\class}{MATH 365}
\begin{document}
\large
\vspace{0.1in}
\noindent\makebox[3.0truein][l]{\textbf{\class}}
\textbf{Name:} \hrulefill \\
\noindent \makebox[3.0truein][l]{\textbf{In class work \quiz, \term \/ \the\year}}
\textbf{Row and Seat}:\hrulefill\\
\vspace{0.1in}


%\noindent  In class work  \quiz\/  has questions 1 through  \numquestions \/ with a total of  \numpoints\/  points.   
%Turn in your work at the end of class  \emph{on paper}. This assignment is due \emph{\due}.

\vspace{0.1in}


\begin{questions} 

  \question  Find \emph{both} square roots of $-1 + \imag$. Express your answer in rectangular form involving nested square roots.  (AA) 
  
  \question Find the \emph{principal square root} of $-1 + \imag$. Express your answer in rectangular form involving nested square roots.
  (SB) 
  
  \question Find the \emph{principal square root} of $\mathrm{e}^{18 \imag}$. Express your answer in rectangular form involving
  trigonometric functions.  \textbf{Hint:}  You might like to use the fact that (DJ)
  \[
         \Arg(z) = \arg(z) - 2 \uppi \left \lceil \frac{\arg(z)  -  \uppi}{2 \uppi} \right \rceil.
  \]
  
  \question Let $\sqrt{\phantom{x}}$ be the \emph{principal square root}. Show that $\sqrt{\overline{z}} = \overline{\sqrt{z}} $ isn't an identity.
  \textbf{Hint:}  Try $z=-1$.  (AK)
  
    \question Let $\sqrt{\phantom{x}}$ be the \emph{principal square root}. Show that $\sqrt{z^2} = z $ isn't an identity.
  \textbf{Hint:}  Try $z=-1$.   (CR)
  
   
    \question Let $\sqrt{\phantom{x}}$ be the \emph{principal square root}. Show that $\sqrt{z w} = \sqrt{z} \sqrt{w} $ isn't an identity.
  \textbf{Hint:}  Try $z=-1$ and $w=-1$.  (MS)
  
  
  
\end{questions}


\end{document}

